\section{Discussion}
\label{sec:discussion}

\subsection{Biases}

It is difficult to draw a truly uniform sample of Tor users.  The only way to
reach all Tor users uniformly would be to modify Tor Browser's landing page
that is displayed on start---an approach that we considered prohibitively
invasive.  Instead, we decided to ask The Tor Project to disseminate our survey
on its blog and social media accounts.  We believe that this recruitment
strategy was subject to the following biases.

\paragraph{Non-response bias.} 
People who noticed our call for volunteers and decided against participating
may exhibit traits that are fundamentally different from those who did
participate.  These non-respondents may have valued their privacy too much,
falsely believed that their experience is irrelevant, lacked the time, or had
other reasons not to participate.

\paragraph{Survivor bias.}
We predominantly heard from people who can tolerate Tor's usability issues,
which is why they are still around to tell their tale.  We likely did not hear
from many---if any---people who gave up on Tor and were thus unable to tell us
what drove them away.  The danger of survivor bias lies in optimizing the user
experience for the subset of people who can tolerate a non-optimal user
experience.

\paragraph{Self-selection bias.}
Due to the nature of our online survey, participants could voluntarily select
themselves into the group of respondents.  This set of people may be unusually
engaged and technical, which is why they have formed opinions that they
consider worth sharing.

\subsection{Take-aways for Tor developers}

Several of our interview participants pointed out Tor Browser's antiquated user
interface.  Past work has shown that users interpret unrelated aspects such as
voice quality as a proxy signal for security, which raises the question if the
same holds true for user interface design~\cite[\S~IV.A]{Abu-Salma2017a}.  If
so, it is important to equip Tor Browser with a modern user interface.

\begin{figure}[t]
    \centering
    \includegraphics[width=\linewidth]{figures/tor-comic.jpg}
    \caption{A comic draft that illustrates what Tor can and cannot provide for
        non-technical users.  The comic was drawn by artist Jason Li while
        working with one of the authors.}
    \label{fig:tor-comic}
\end{figure}

\subsection{Take-aways for Tor users}

The strong security properties of onion services are futile if users cannot tell
apart a genuine domain from its impersonation.  Awareness of this issue is the
first step and several onion services have long begun to alert their users.



% could tor show you when an onion service descriptor was first published?


\subsection{Take-aways for Tor researchers}

We found it challenging, yet rewarding and illuminating to study the Tor
community.  Tor users obviously value their privacy which reduces their
willingness to participate in research projects.  Past academic research
projects that involved questionable methods turned this care into distrust for
many users.  Showing willingness to directly interact with the community and
taking seriously their concerns signals respect and transparent methods.  For
our online survey, we recommend to use software that works in Tor
Browser\footnote{Note that Tor Browser supports three security levels; the
default of ``low,'' ``medium,'' and ``high.''  Some users brought to our
attention that our survey did not work when the security level is set to
``high'' because it disables JavaScript, which our survey required---an
oversight on our end.} and does not fetch tracking scripts such as Google
Analytics.  For the survey design, one likely has to forego asking questions
that are best practice such as income level and country of residence.  We made
even the basic information we asked optional so our respondents had the chance
to answer the survey without providing any personal information at all.  In our
interviews we tried to accomodate the needs of our participants by using
software of their choice.
