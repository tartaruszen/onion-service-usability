\section{Discussion}
\label{sec:discussion}

We now reflect on our work by elaborating on its shortcomings
(\Cref{sec:biases}) and proposing future research directions
(\Cref{sec:future-work}).

\subsection{Biases}
\label{sec:biases}

A truly uniform sample of Tor users is difficult to obtain.  One strategy would
have been to work with The Tor Project to add a link to our experiment on Tor
Browser's landing page.  While this approach would have reached a large number
of Tor users, we considered it prohibitively invasive.  Besides, people who
rarely restart their Tor Browser or pay no attention to the landing page would
have still missed our experiment.  We therefore decided to ask The Tor Project
to disseminate our survey on its blog and Twitter account.  We believe that this
recruitment strategy was subject to the following biases.

\paragraph{Non-response bias.}
People who noticed our call for volunteers but decided against participating may
have valued their privacy too much, falsely believed that their perspective is
irrelevant, lacked time, or had other reasons not to participate.  Nevertheless,
non-respondents may exhibit traits that are fundamentally different from those
who did participate, which is why their absence in our sample may bias our
results.

\paragraph{Survivor bias.}
We predominantly heard from people who can tolerate Tor's usability issues,
which is why they are still around to tell their tale.  We likely did not hear
from many---if any---people who decided that Tor Browser was not for them, and
were thus unable to tell us what drove them away.  The danger of survivor bias
lies in optimizing the user experience for the subset of people whose tolerance
for inconvenience is higher than the rest.

\paragraph{Self-selection bias.}
Due to the nature of our online survey, participants could voluntarily select
themselves into the group of respondents.  This set of people may be unusually
engaged and technical, which is why they have formed opinions that they
consider worth sharing.  Indeed, the demographic for our online survey in
\Cref{sec:online-survey} was rather young and educated.

\subsection{Future work}
\label{sec:future-work}

The Tor Project is currently working on a security indicator for onion
services~\cite{trac23247}.  Recall that the current version of Tor Browser (see
\Cref{fig:onion-service}) displays an onion service like a plain HTTP
connection, which greatly ``undersells'' the security and privacy that an onion
service connection provides.  Porter Felt \ea~\cite{Felt2016a} showed that many
subtleties must be considered when designing security indicators which is why an
evaluation of The Tor Project's design would be of great value.  For example, do
users (correctly) understand what an onion service indicator means and what it
provides on top of an HTTPS indicator?

Ensuring that users understand technology correctly does not stop at security
indicators.  Another one of Tor Browser's striking UI deviations from Firefox is
its Tor circuit display (see \Cref{fig:tor-button}).  Some users are not
familiar with the concept of guard relays and incorrectly expect each relay in
their circuit to change.

Going forward, we believe that there is a need for a system that automates onion
service discovery, \eg, by providing an opt-in mechanism that automatically
publishes onion domains in a public log.  Users can then monitor this log and
learn about new onion domains as they get published.

An orthogonal issue is transparent redirection to onion domains, which we
discussed in \Cref{sec:onion-redirection}.  Clearly, there is a need for some
kind of redirection mechanism.  The Tor Project is already experimenting with
prototypes.

Finally, our survey responses suggest that there is a need for a
privacy-preserving bookmarking tool that allows Tor Browser users to bookmark
sites without leaving a browsing trail on their hard drives.
