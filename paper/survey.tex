\section{Online survey}
\label{sec:online-survey}

\subsection{Survey design}
\label{sec:survey-design}

We investigate how Tor users interact with onion services by designing and
administering a survey.

\subsubsection{Research question}
We designed the survey to answer the following research questions: \emph{How do
Tor users interact with onion services?}  An answer to this research question
allows us to both create more usable anonymity systems and build these systems
in a way that human-centered attacks such as phishing are exacerbated.  In
particular, we seek to answer the following three aspects:

\begin{itemize}
    \item What is the expectation of privacy when people use Tor Browser in
        general and onion sites in particular?
    \item What is the security/usability trade-off of vanity onion domains?
    \item Do people handle onion domains differently than normal domains?
\end{itemize}

\subsubsection{Design}

\begin{itemize}
    \item Our survey consists of six blocks:
        \begin{enumerate}
            \item Consent and demographics
            \item Tor usage
            \item Onion site usage
            \item Onion site operation
            \item Onion site phishing and impersonation
            \item Expectations of privacy
        \end{enumerate}
    \item We implemented our survey in Qualtrics and made sure that it can be
        answered correctly over Tor Browser---it needed JavaScript, though.
        Used a Qualtrics feature so that participants can answer the survey only
        once.
    \item We used four screener questions distributed over four distinct
        blocks, \ie, questions whose sole purpose is to check whether the
        respondent is paying attention.~\cite{Berinsky2014a}.
    \item Having followed mailing lists \etc for many years, we did not feel the
        need for focus groups to explore what topics are worth inquiring.
    \item Used cognitive pretesting (sometimes also called cognitive
        interviewing)~\cite{Collins2003a}.  Pretesting can show us that
        respondents \first understand questions, \second understand questions
        consistently, and \third understand questions the way that we intended.
        Two main strategies are \emph{think-aloud interviewing} and
        \emph{probing}.  In addition, we can ask respondends about the
        confidence they have in their responses.  However, not all cognitive
        processes can be verbalized and cognitive pretesting may change the way
        respondents answer questions.  We had $N$ respondents (talk about
        demographics) based on whose input we iteratively improved our survey.
        $M$ respondents were fluent, but non-native English speakers.
    \item We tried hard to have neutral, non-leading questions.
    \item Basic statistics.  The survey consists of X questions and takes
        approximately Y minutes to complete.  We used our university's
        Qualtrics membership to create the survey.
\end{itemize}

\subsubsection{Participant recruiting}
\begin{itemize}
    \item Tor's population is unknown.  We barely know its size.  Besides, there
        is no way to recruit all Tor users with equal probability.  Therefore,
        we will have inevitable sampling bias in our results.  We work around
        that by using diverse recruitment media and making our sample size as
        large as possible.
    \item Post on Tor's blog\\\url{https://blog.torproject.org/blog/take-part-study-help-improve-onion-services}.
    \item Tor tweeted it\\\url{https://twitter.com/torproject}
    \item Post on social media such as:
        \begin{itemize}
            \item \url{https://reddit.com/r/tor/}
            \item \url{https://reddit.com/r/onions/}
            \item \url{https://reddit.com/r/samplesize/}
        \end{itemize}
    \item Started survey on 2017-08-16 and ended survey on 2017-XX-XX.
\end{itemize}

\begin{itemize}
    \item Are our results generalizable to self-authenticating names?
    \item Use aided recall for behavioral questions.
    \item Questions should be non-threatening (is there a ``right'' or ``wrong''
        answer?).  If respondents think they would
        look bad, they may answer not truthfully.
    \item Avoid jargon and unusual vocabulary (to non-native English speakers)
\end{itemize}

\subsubsection{Incentives for participation}
\begin{itemize}
    \item Difficult to do in anonymous fashion so we decided to not have
        incentives for survey.
\end{itemize}

\subsubsection{Data analysis}

\subsubsection{Limitations}
\begin{itemize}
    \item Representative sample?
\end{itemize}

\subsection{Results}
\label{sec:results}

\subsubsection{Data pruning}
\begin{itemize}
    \item Median response time was 15 minutes, but wide variance.
    \item Attention checks didn't work that well.
\end{itemize}

\subsubsection{Biases}
\begin{itemize}
    \item Security-conscious people didn't respond because our survey required
        JavaScript.
    \item Self-selection bias.
    \item We reached mostly heavy Tor users who presumably know more than the
        occasional user.
    \item Survivor bias: We heard mostly/only from people who can tolerate Tor's
        usability issues.
\end{itemize}

\subsubsection{Data quality}
\begin{itemize}
    \item Attention checks.
\end{itemize}

\subsubsection{Demographics}
Table~\ref{tab:survey-demo} shows the demographics of our survey.  Not
surprisingly, our demographic is \emph{young and educated}: more than sixty
percent of respondents are younger than 36, and another sixty percent has at
least a graduate degree.  Finally, another sixty percent considers themselves
at least highly knowledgeable in matters of Internet privacy and security.

\begin{table*}[ht]
	\centering
	\begin{tabular}{l r l r l r l r}
	\toprule
	Gender & \% &
	Age & \% &
	Education & \% &
	Domain knowledge & \% \\
	\midrule
	Female & 8.9  & 18--25   & 35.5 & No degree     & 5.5  & No knowledge             & 0.5  \\
	Male   & 86.3 & 26--35   & 34.6 & High school   & 31.9 & Mildly knowledgeable     & 7.6  \\
	Other  & 4.8  & 36--45   & 17.5 & Graduate      & 42.3 & Moderately knowledgeable & 32.4 \\
	       &      & 46--55   & 8.7  & Post graduate & 20.4 & Highly knowledgeable     & 44.9 \\
	       &      & 56--65   & 2.5  &               &      & Expert                   & 14.6 \\
	       &      & $>$ 65   & 1.2  &               &      & & \\
	\bottomrule
	\end{tabular}
	\caption{The distribution over gender, age, education, and domain knowledge 
	for our 621 interview subjects.}
	\label{tab:survey-demo}
\end{table*}

\subsubsection{Tor usage}
The three entities most Tor users seek to protect themselves against are
governments (X\%), ISPs (X\%), and corporations (X\%).  Generally speaking,
people use Tor to protect themselves from a wide variety of threats including:
\begin{description}
    \item[Corporations] because of tracking, advertising, and profiling.  Some
        respondents specifically pointed out Google and Facebook.
    \item[Service providers] such as ISPs, backbone ISPs, and websites
        themselves.  In fact, the ISP is the most popular ``threat'' to our
        respondents with XX\%.
    \item[Governments] because of mass surveillance, law enforcement, and
        government-mandated censorship.  Governments are the second most
        popular threat with XX\%.
    \item[Personal threats] including identity theft, targeted harassment, and
        stalking.
    \item[Research] allows users to learn about a topic without revealing their
        interest in it.  Some respondents use Tor for search engine
        optimization, computer security research, and to research medical
        conditions.
    \item[Technical] use cases include IPv6 connectivity, the evasion of
        geographical restrictions, and access to onion services.  A small
        number of respondents is only interested in technical aspects other
        than privacy.
\end{description}

One respondent stated that they don't need anonymity themselves but instead use
Tor to provide cover traffic for ``people who need protection.''

Another respondent uses Tor to have IPv6 connectivity---some exit relays are
multi-homed and can connect to both IPv4 and IPv6 endpoints.

Finally, some respondents mentioned that they are using Tor only to connect to
onion services.

Half of our respondents use Tor either once a day or consider Tor Browser their
main browser.

\subsubsection{Onion service usage}

\subsubsection{Perception of next-generation domain format}

Recall that the next generation of onion services will extend the domain length
from 16 to 54 characters, almost three times the length of the current domain
format.

We asked our survey respondents if they expect this format change to affect
their browsing habits.  591 users answered this question.  95 (16\%) selected
that prop224 domains will change their habits while the remaining 496 (84\%)
selected that their habits won't be affected.

Respondents who believe that their habits will change (16\%) gave the
following reasons:

These results suggest that the new domain format is among the
minor usability issues surrounding onion services.  In fact, an
easy-to-remember domain format ranks last among the six criteria whose
importance we asked users about.  On a five-point Likert scale ranging
from "not at all important" to "very important," we got the following
results:

Problematic:
``I only memorize the first part of the domain''

``If there isn't some cognizable word at the start, it'll be more difficult for
me to determine if I'm going to the correct domain or a scam. I may end up going
to less .onion sites as a result.''

\subsubsection{Susceptibility to phishing}
\begin{itemize}
    \item Attack partially documented in academic
        literature~\cite[\S~5.1]{Winter2016a}.  Monteiro provided a
        timeline~\cite{Monteiro2016a}.
    \item Many ad-hoc anti-phishing methods, some of which incorrect.
\end{itemize}

\subsubsection{Perception of vanity onion domains}
\begin{description}
    \item[$+$] Combination of vanity prefix and Markov Model-selected suffix can
        deliver memorable domain for v2 domains.
    \item[$+$] Reveals what the onion service is hosting.
    \item[$-$] May provide false sense of security.
    \item[$-$] Only available to people with resources.
    \item[$-$] Conflicts with peoples' (flawed) expectation that a domain must
        be random to stay hidden.
\end{description}

Technical ways to exacerbate creation of vanity domains: Incorporate scrypt into
key generation process.
If vanity domains remain, have Tor process support it?

\subsubsection{Participants}
\begin{itemize}
    \item Provide intuition on how much our sample resembles the general
        population.
    \item From when to when did we disseminate our survey?
    \item How many participants did we attract?
    \item How long did it take to complete the survey?  Provide some descriptive statistics.
    \item What's the female/male ratio?
    \item What's the age distribution?
    \item What's the level of education?
    \item What's the computer security knowledge?  Note that our participants
        may overestimate their knowledge.
\end{itemize}

\subsubsection{Random anecdotes}
\begin{itemize}
    \item In pre-test: one person considered the onion domain itself
        confidential and was worried about the domain's prefix revealing
        anything about its content.
\end{itemize}
