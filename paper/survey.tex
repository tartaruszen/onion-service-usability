\section{Online survey}
\label{sec:online-survey}

Approximately one month after we conducted our first batch of interviews, we
launched an online survey.  After having begun to explore the topic in our
interviews, the purpose of this survey was to obtain a large and diverse number
of responses for a number of specific questions.  We incorporated some of the
preliminary interview data in our survey questions.

\subsection{Method}
\label{sec:survey-design}

We created our survey in Qualtrics because our institution had a subscription
and it came with all the features that we deemed necessary.  We verified that an
out-of-the-box Tor Browser could correctly take the survey.  However, Qualtrics
requires JavaScript, which is deactivated when Tor Browser is set to its highest
security setting.  A number of users complained about the reliance on JavaScript
in the comments of our recruitment blog post~\cite{Winter2017a}.

Our survey was only available in English but we targeted an international
audience because there are cultural differences in security
behavior~\cite{Sawaya2017a}.  Ignoring these differences would cause us to
optimize Tor Browser's user experience for a predominantly Western audience,
which is problematic.

While we were developing the survey, we used cognitive pretesting (sometimes
also called cognitive interviewing) to improve the wording of our
questions~\cite{Collins2003a}.  Pretesting reveals if respondents \first
understand questions, \second understand questions consistently, and \third
understand questions the way that we intended.  In our pretests we administered
our survey and asked the respondents to answer it will verbalizing their thought
process.  We occasionally asked follow-up questions to make sure that our
pretesters understood the questions as intended.  However, not all cognitive
processes can be verbalized and cognitive pretesting may change the way
respondents answer questions.  We had five pretesters based on whose input we
iteratively improved our survey.  Two pretesters were native English speakers
while the remaining three were fluent but spoke English as a second language.

The majority of our survey focused on onion services but we also added some
questions about Tor in general.  Table~\ref{tab:survey-structure} shows that our
survey consists of six blocks that are ordered by topic.  It takes about fifteen
minutes to answer all questions.  The full survey is listed in
Appendix~\ref{app:interview-questions}.  Finally, we launched our survey on
August 16, 2017 and ended it on September 11, 2017, so it was active for 27
days.

\begin{table}[t]
	\centering
	\begin{tabular}{l r}
	\toprule
	Topic & \# of questions \\
	\midrule
	Consent and demographic information & 1 \\
	Tor usage & 4 \\
	Onion site usage & 20 \\
	Onion site operation & 5 \\
	Onion site phishing and impersonation & 9 \\
	Expectations of privacy & 9 \\
	End of survey & 1 \\
	\bottomrule
	\end{tabular}
	\caption{The topical question blocks in our survey and the number of
	questions they contain.}
	\label{tab:survey-structure}
\end{table}

\subsubsection{Recruitment}

Similar to our interviews, we advertised our survey in a blog post on The Tor
Project's blog~\cite{Winter2017a}, on its corresponding Twitter account, and on
three Reddit subforums.\footnote{The forums are \url{https://reddit.com/r/tor/},
\url{https://reddit.com/r/onions/}, and \url{https://reddit.com/r/samplesize/}.}
Again, note that this recruitment strategy is likely to bias our sample towards
more enganged users as casual Tor users are unlikely to follow The Tor Project's
blog or Twitter account.

\subsubsection{Research ethics}
Respondents had to agree to a consent form before starting the survey. The
consent form informed the respondents about the procedure of our experiment and
verified that all respondents were at least eighteen years of age.

To provide additional incentives, we originally planned to give respondents the
option to participate in a gift card lottery.  We abandoned the idea because it
was non-trivial to reconcile a lottery with anonymous participation because we
would have to collect our respondents' email addresses.  Despite the lack of
incentives, we collected a sufficient number of responses.  In fact, we believe
that many of our respondents were primarily motivated by improving Tor---some of
our interview participants turned down the gift cards that we offered.

\subsection{Results}
\label{sec:results}

\subsubsection{Demographics}

\subsubsection{Data pruning}
\begin{itemize}
    \item Median response time was 15 minutes, but wide variance.
    \item Attention checks didn't work that well.
    \item We used four screener questions distributed over four distinct
        blocks, \ie, questions whose sole purpose is to check whether the
        respondent is paying attention.~\cite{Berinsky2014a}.
\end{itemize}

\subsubsection{Biases}
\begin{itemize}
    \item Security-conscious people didn't respond because our survey required
        JavaScript.
    \item Self-selection bias.
    \item We reached mostly heavy Tor users who presumably know more than the
        occasional user.
    \item Survivor bias: We heard mostly/only from people who can tolerate Tor's
        usability issues.
\end{itemize}

\subsubsection{Data quality}
\begin{itemize}
    \item Attention checks.
\end{itemize}

\subsubsection{Demographics}
Table~\ref{tab:survey-demo} shows the demographics of our survey.  Not
surprisingly, our demographic is \emph{young and educated}: more than sixty
percent of respondents are younger than 36, and another sixty percent has at
least a graduate degree.  Finally, another sixty percent considers themselves
at least highly knowledgeable in matters of Internet privacy and security.

\begin{table*}[ht]
	\centering
	\begin{tabular}{l r l r l r l r}
	\toprule
	Gender & \% &
	Age & \% &
	Education & \% &
	Domain knowledge & \% \\
	\midrule
	Female & 8.9  & 18--25   & 35.5 & No degree     & 5.5  & No knowledge             & 0.5  \\
	Male   & 86.3 & 26--35   & 34.6 & High school   & 31.9 & Mildly knowledgeable     & 7.6  \\
	Other  & 4.8  & 36--45   & 17.5 & Graduate      & 42.3 & Moderately knowledgeable & 32.4 \\
	       &      & 46--55   & 8.7  & Post graduate & 20.4 & Highly knowledgeable     & 44.9 \\
	       &      & 56--65   & 2.5  &               &      & Expert                   & 14.6 \\
	       &      & $>$ 65   & 1.2  &               &      & & \\
	\bottomrule
	\end{tabular}
	\caption{The distribution over gender, age, education, and domain knowledge 
	for our 621 interview subjects.}
	\label{tab:survey-demo}
\end{table*}

\subsubsection{Tor usage}
The three entities most Tor users seek to protect themselves against are
governments (X\%), ISPs (X\%), and corporations (X\%).  Generally speaking,
people use Tor to protect themselves from a wide variety of threats including:
\begin{description}
    \item[Corporations] because of tracking, advertising, and profiling.  Some
        respondents specifically pointed out Google and Facebook.
    \item[Service providers] such as ISPs, backbone ISPs, and websites
        themselves.  In fact, the ISP is the most popular ``threat'' to our
        respondents with XX\%.
    \item[Governments] because of mass surveillance, law enforcement, and
        government-mandated censorship.  Governments are the second most
        popular threat with XX\%.
    \item[Personal threats] including identity theft, targeted harassment, and
        stalking.
    \item[Research] allows users to learn about a topic without revealing their
        interest in it.  Some respondents use Tor for search engine
        optimization, computer security research, and to research medical
        conditions.
    \item[Technical] use cases include IPv6 connectivity, the evasion of
        geographical restrictions, and access to onion services.  A small
        number of respondents is only interested in technical aspects other
        than privacy.
\end{description}

One respondent stated that they don't need anonymity themselves but instead use
Tor to provide cover traffic for ``people who need protection.''

Another respondent uses Tor to have IPv6 connectivity---some exit relays are
multi-homed and can connect to both IPv4 and IPv6 endpoints.

Finally, some respondents mentioned that they are using Tor only to connect to
onion services.

Half of our respondents use Tor either once a day or consider Tor Browser their
main browser.

\subsubsection{Onion service usage}

\subsubsection{Perception of next-generation domain format}

Recall that the next generation of onion services will extend the domain length
from 16 to 54 characters, almost three times the length of the current domain
format.

We asked our survey respondents if they expect this format change to affect
their browsing habits.  591 users answered this question.  95 (16\%) selected
that prop224 domains will change their habits while the remaining 496 (84\%)
selected that their habits won't be affected.

Respondents who believe that their habits will change (16\%) gave the
following reasons:

These results suggest that the new domain format is among the
minor usability issues surrounding onion services.  In fact, an
easy-to-remember domain format ranks last among the six criteria whose
importance we asked users about.  On a five-point Likert scale ranging
from "not at all important" to "very important," we got the following
results:

Problematic:
``I only memorize the first part of the domain''

``If there isn't some cognizable word at the start, it'll be more difficult for
me to determine if I'm going to the correct domain or a scam. I may end up going
to less .onion sites as a result.''

\subsubsection{Susceptibility to phishing}
\begin{itemize}
    \item Attack partially documented in academic
        literature~\cite[\S~5.1]{Winter2016a}.  Monteiro provided a
        timeline~\cite{Monteiro2016a}.
    \item Many ad-hoc anti-phishing methods, some of which incorrect.
\end{itemize}

\subsubsection{Perception of vanity onion domains}
\begin{description}
    \item[$+$] Combination of vanity prefix and Markov Model-selected suffix can
        deliver memorable domain for v2 domains.
    \item[$+$] Reveals what the onion service is hosting.
    \item[$-$] May provide false sense of security.
    \item[$-$] Only available to people with resources.
    \item[$-$] Conflicts with peoples' (flawed) expectation that a domain must
        be random to stay hidden.
\end{description}

Technical ways to exacerbate creation of vanity domains: Incorporate scrypt into
key generation process.
If vanity domains remain, have Tor process support it?

\subsubsection{Participants}
\begin{itemize}
    \item Provide intuition on how much our sample resembles the general
        population.
    \item From when to when did we disseminate our survey?
    \item How many participants did we attract?
    \item How long did it take to complete the survey?  Provide some descriptive statistics.
    \item What's the female/male ratio?
    \item What's the age distribution?
    \item What's the level of education?
    \item What's the computer security knowledge?  Note that our participants
        may overestimate their knowledge.
\end{itemize}

\subsubsection{Random anecdotes}
\begin{itemize}
    \item In pre-test: one person considered the onion domain itself
        confidential and was worried about the domain's prefix revealing
        anything about its content.
\end{itemize}
