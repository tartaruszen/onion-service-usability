\subsection*{Abstract}
% Problem statement.
Onion services are anonymous TCP services that are exposed over the Tor network.
As of November 2017 more than 50,000 onion services make available web sites,
chat protocols, and file sharing services.  Compared to conventional Internet
services, onion services exhibit higher latency; can only be accessed over Tor;
are barely indexed by search engines; and employ long, self-authenticating
domain names.  Our understanding of how users deal with these idiosyncrasies is
anecdotal.
% What we are doing.
In this work we fill this gap by studying how people perceive, understand, and
use onion services.  To that end we use a mixed-methods approach consisting of
semi-structured interviews to explore the problem space, and an online survey to
solicit answers to concrete questions.  We find that \first users place great
trust in The Tor Project but distrust content that is hosted on onion services,
\second users have devised diverse methods to work around the non-memorable
domain format, \third some users have flawed mental models of the underlying
technology, and \fourth users not only rely on onion services for anonymity but
also for their security and NAT-punching properties.
% Why our work matters.
Our work enables The Tor Project to focus its efforts on the most pressing
usability issues, address common misconceptions by improving its documentation.
