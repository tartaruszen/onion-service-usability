\section{Introduction}
\label{sec:introduction}

The colloquial meaning behind online anonymity typically implies client
anonymity, \ie, a user disguises her IP address towards a web service, for
example by using a VPN.  A lesser-known use case is server anonymity in which
the IP address of a web service is disguised.  Just like end users, server
operators can have good reasons to disguise the location of their services; be
it to escape harassment, evade legal repercussions, or voice dissenting
opinions, just to name a few.  Perhaps the most popular way of exposing an
anonymous service on the Internet is Tor's onion services.\footnote{Onion
services used to be known as ``hidden services'' but were recently relabeled to
reflect the fact that onion services provide more than just ``hiding'' a
service.}

Originally deployed in 2004, onion services have grown substantially over the
last years, both in the number of services and users.  As of May 2017, The Tor
Project's statistics show that more than 50,000 onion services are online each
day, relaying more than 750 MBps of network traffic.  Not all of these services
host web sites: other use cases such as metadata-free
messaging~\cite{ricochet} and file sharing~\cite{onionshare} have emerged as
well.  Learning the number of onion service users is more challenging.  In 2016,
Facebook reported that more than one million users logged into their onion
service over a one month period~\cite{facebook-users}.

% Enumerate usability issues in the form of a short experience story instead?
% start with discover, then that you need Tor, then weird domain format, etc
From a usability point of view, onion services differ from conventional web
services in several aspects: \first they can only be accessed over the Tor
network; \second their domains are hashes over public keys, rendering them hard
to remember; \third network latency is noticeable because of the additional hops
in between client and onion service; and \fourth onion services are private by
default and must be manually disseminated by their operator. To date our
understanding of how users deal with these idiosyncrasies is merely anecdotal.
Despite the lack of literature, much work has investigated potential
improvements.  In this work, we seek to fill this gap by answering the research
question of how Tor users interact with onion services.  In particular, we set
out to understand users' mental models, their expectations of privacy, the issues
that they experience, and how they adapted their workflows to deal with onion
services.

Now, onion services don't exist in a vacuum.  They are tightly coupled to their
surrounding software ecosystem, most importantly Tor Browser.  Focusing our work
on onion services only would miss important context.  While our research
question is about onion services, we aim to get as complete a picture as
possible by casting a wider net and also answering open questions about the use
of Tor Browser in particular and privacy expectations in general.

This work seeks to fill this gap by tackling the research questions of \emph{how
Tor users interact with onion services?}  To that end, we employ a mixed-methods
approach.  We created an online survey that asked participants to answer an
array of questions on Tor Browser, onion service usage and operation, onion site
phishing, and general expectations of privacy.  A survey prevents us from
reacting to our participants, which is why we also conducted a series of
semi-structured interviews to explore the problem space outside the constraints
of our online survey.  Based on our survey questions, our interviews let us ask
follow-up questions and dive deeper into unexpected answers.

Our most salient findings show that \first many users have flawed mental models
of Tor that may cause security issues, \second the domain format of onion
services, while cumbersome, is not among the most pressing usability issues,
\third the content that onion services are perceived to host causes trust issues
for non-technical users, and \fourth onion service operators fight phishing
attacks by having devised a number of strategies, many of which are ineffective.

Our work benefits The Tor Project by serving as guidelines to improve the
usability of both Tor Browser and onion services.  At the time of this writing,
The Tor Project is finalizing the design and implementation of the next
generation of onion services, and we hope that our results can inform this
process.  Finally, as many of our findings touch on various aspects of privacy
and anonymity, we hope that efforts beyond the Tor network can benefit.  In
summary, we make the following contributions:

\begin{itemize}
    \item We interviewed seventeen participants of diverse backgrounds to
        understand their habits, assumptions, and expectations related to Tor.
        Using qualitative data coding, we identified novel themes that underlay
        our interview data.

    \item We administered an online survey for Tor users, reaching 621 people.
        Our survey focused on the most salient issues around onion services,
        \eg, the domain format, onion service discovery, and phishing attacks.

    \item Drawing on our two datasets, we identify \first incorrect mental
        models, \second key issues that impede the adoption and use of onion
        services, and \third ways forward to improve onion services.
\end{itemize}

The rest of this paper is structured as follows.  We begin by discussing related
work in Section~\ref{sec:related-work}, followed by background on onion services
in Section~\ref{sec:background}.  Section~\ref{sec:interview-study} then dives
into our interview study, followed by Section~\ref{sec:online-survey} which
focuses on our online survey.  Finally, we discuss our results in
Section~\ref{sec:discussion} and conclude our work in
Section~\ref{sec:conclusion}.
