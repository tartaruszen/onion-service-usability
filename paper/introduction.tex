\section{Introduction}
\label{sec:introduction}
The Tor Project's onion services provide what may be the
most popular way of running an anonymous \textsc{tcp} service. Typically, online anonymity implies \emph{client} anonymity
such as the use of a \textsc{virtual private network} to disguise one's \textsc{ip} address.  However, Tor onion services employ a lesser-known use case of anonymity, that is \emph{server} anonymity which allows a web service to
disguise its \textsc{ip} address.  Service operators have good reasons to employ
server anonymity; be it to escape harassment, speak out against power, or voice
dissenting opinions.  \footnote{Onion
services were originally called ``hidden services'' but were recently renamed to
reflect the fact that onion services provide more than just ``hiding'' a
service~\cite{Johnson2015a}---most importantly end-to-end security and
self-authenticating names.}

Originally deployed in 2004, onion services have grown substantially over the
last years, both in the number of servers and users.  As of February 2018, The
Tor Project's statistics count more than 60,000 onion services each day,
relaying an aggregate of 1 Gbps of network traffic.  Not all of these services
host web sites; use cases such as metadata-free instant
messaging~\cite{ricochet} and file sharing~\cite{onionshare} have emerged as
well.  The Tor Project currently does not have data on the number of onion
service users but Facebook reported in 2016 that more than one million users
logged into their onion service over a one-month period~\cite{facebook-users}.

Onion services differ from conventional web services in several aspects;
\first~they can only be accessed over the Tor network; \second~onion domains are
hashes over their public key, rendering them hard to remember; \third~network
latency is noticeable because of the additional hops in between client and the
onion service; and \fourth~onion services are private by default, requiring
manual dissemination.  To date, our understanding of how users deal with these
idiosyncrasies is anecdotal.  We fill this gap by studying how Tor users
interact with onion services.

However, onion services do not exist in a vacuum.
They are tightly coupled to their
surrounding software ecosystem---most importantly, the Tor Browser---which is why an
isolated study of onion services without accounting for Tor is bound to miss crucial context.  Therefore, we set out to understand users'
mental models of onion services \emph{and} the Tor network where applicable, and we study 
how they use and manage onion services, the challenges and benefits of using onion services, and how they adapt their workflow to deal with onion services.

To that end, we employ a mixed-methods approach. First, we conducted exploratory interviews with Tor and onion service users to guide the design of an online survey. We then conducted a large scale online survey that featured an array of questions on Tor Browser, onion service usage and operation, onion site
phishing, and users' general expectations of privacy. Finally, we conducted follow up interviews to flesh out the topics uncovered in the exploratory interviews and the survey to allow us a deeper dive into the themes of interest.

Our findings include that \first~many Tor users misunderstand technical aspects
of onion services such as the nature of the domain format, rendering these users
more vulnerable to phishing attacks; \second~users handle the pseudo-random
onion domain format differently, but mostly by bookmarking them; and \third~the
way people discover onion services is provisional and in need of technical
enhancements to ease the process.

At the time of this writing, The Tor Project is testing the next generation of
onion services intended to fix security issues and upgrade to faster and
future-proof cryptography.  Our results can inform this process.
Finally, because some of our findings touch on general aspects of privacy and
anonymity, we believe that privacy engineers beyond The Tor Project can benefit
from our findings.  In summary, we make the following contributions:

\begin{itemize}
    \item We interviewed seventeen Tor users of diverse backgrounds to
        understand their issues, assumptions, and expectations related to the
        Tor network.  Using qualitative data coding, we identified and present
        themes and anecdotes that underlay our interview data.

    \item We administered an online survey for Tor users, gathering 527 full
        responses.  Our survey focused on the most salient issues around onion
        services, \ie, the domain format, onion service discovery, and phishing
        attacks.

    \item Drawing on our two datasets, we identify key issues that impede the
        adoption of onion services and propose ways forward, \eg, an opt-in
        publishing mechanism for onion services, and a Tor Browser extension
        that allows its users to securely bookmark onion domains without
        exposing a browsing trail.
\end{itemize}

The rest of this paper is structured as follows.  We begin by discussing related
work in \Cref{sec:related-work}, followed by background on onion services in
\Cref{sec:background}.  \Cref{sec:method} then presents the methods we used for
our interviews and online survey, followed by \Cref{sec:results} which discusses
our findings from both data sources.  Finally, we discuss our work in
\Cref{sec:discussion} and conclude in \Cref{sec:conclusion}.
