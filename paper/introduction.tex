\section{Introduction}
\label{sec:introduction}

The colloquial meaning behind online anonymity implies \emph{client anonymity},
\ie, a user disguises her IP address, for example by using a VPN.  A
lesser-known use case is \emph{server anonymity} which allows a web service to
disguise its IP address.  Service operators have good reasons to employ server
anonymity; be it to escape harassment, speak out against power, or voice
dissenting opinions.  Tor's onion services provide what may be the most popular
way of running an anonymous TCP service.\footnote{Onion services used to be
known as ``hidden services'' but were recently renamed to reflect the fact
that onion services provide more than just ``hiding'' a service---most
importantly end-to-end security and self-authenticating names.}

Originally deployed in 2004, onion services have grown substantially over the
last years, both in the number of services and users.  As of January 2018, The
Tor Project's statistics count more than 60,000 onion services each day,
relaying an aggregate of more than 750 Mbps of network traffic.  Not all of
these services host web sites; use cases such as metadata-free instant
messaging~\cite{ricochet} and file sharing~\cite{onionshare} have emerged as
well.  The Tor Project currently does not have information on the number of
onion service users but Facebook reported in 2016 that more than one million
users logged into their onion service over a one-month
period~\cite{facebook-users}.

Regarding usability, onion services differ from conventional web services in
several aspects; \first they can only be accessed over the Tor network; \second
their domain is a hash over their public key, rendering them hard to remember;
\third network latency is noticeable because of the additional hops in between
client and onion service; and \fourth onion services are private by default,
requiring manual dissemination.  To date our understanding of how users deal
with these idiosyncrasies is anecdotal.  We fill this gap by studying how Tor
users interact with onion services.  In particular, we set out to understand
users' mental model of Tor, their expectations of privacy, issues that they
experience, and how they adapted their workflows to deal with onion services.

Onion services don't exist in a vacuum.  They are tightly coupled to their
surrounding software ecosystem---most importantly Tor Browser---which is why an
isolated study of onion services is bound to miss important context.  While our
research question is about onion services, we aim to get as complete a picture
as possible by casting a wider net and answering open questions about the use of
Tor Browser in particular and privacy expectations in general.  To that end we
employ a mixed-methods approach involving the creation of an online survey that
asked participants to answer an array of questions on Tor Browser, onion service
usage and operation, onion site phishing, and general expectations of privacy.
Based on our survey questions, our interviews let us ask follow-up questions and
dive deeper into unexpected answers.

Our most salient findings show that \first the flawed mental model some users
have of Tor may cause security issues such as a blind reliance on vanity onion
domains, \second the domain format of onion services, while cumbersome, is not
among the most pressing usability issues, \third the content that onion services
are perceived to host causes trust issues for non-technical users, and \fourth
onion service operators seek to foil phishing attacks by having devised a number
of strategies, many of which are ineffective.

At the time of this writing, The Tor Project is testing the next generation of
onion services, intended to fix security issues and transition to faster and
future-proof cryptography.  We hope that our results can inform this process.
Finally, as many of our findings touch on various aspects of privacy and
anonymity, we believe that systems beyond the Tor network can benefit.  In
summary, we make the following contributions:

\begin{itemize}
    \item We interviewed seventeen Tor users of diverse backgrounds to
        understand their habits, assumptions, and expectations related to Tor.
        Using qualitative data coding, we identified novel themes that underlay
        our interview data.

    \item We administered an online survey for Tor users, reaching 621 people.
        Our survey focused on the most salient issues around onion services,
        \eg, the domain format, onion service discovery, and phishing attacks.

    \item Drawing on our two datasets, we identify \first incorrect mental
        models, \second key issues that impede the adoption and use of onion
        services, and \third ways forward to improve onion services.
\end{itemize}

The rest of this paper is structured as follows.  We begin by discussing related
work in Section~\ref{sec:related-work}, followed by background on onion services
in Section~\ref{sec:background}.  Section~\ref{sec:interviews} then dives into
our interview study, followed by Section~\ref{sec:online-survey} which focuses
on our online survey.  Finally, we discuss our results in
Section~\ref{sec:discussion} and conclude our work in
Section~\ref{sec:conclusion}.
