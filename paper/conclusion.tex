\section{Conclusion}
\label{sec:conclusion}

In this work we studied how Tor users interact with onion services and the
broader technology that enables onion services.  Drawing on a mixed-methods
approach, we conducted seventeen semi-structured interviews and collected 828
responses to our online survey.  These two data sets served as the basis of our
analysis and provided unique insight into how Tor users perceive, use, and
understand Tor in general and onion services in particular.

We find that the current state of onion services resembles the web of the 90s:
Pages load slowly, user interfaces are clumsy, and search engines fall short of
expectations.  Users appreciate the extra security, privacy, and \textsc{nat}
punching properties of onion services, giving rise to numerous use cases, but
are frequently wary of the content that is hosted on them.  Some users found
ways to manage what is perhaps the most striking idiosyncracy of onion
services---their domain format---by using bookmarks, using (encrypted) files
containing links, and referring to trusted link aggregators.  Many users however
have flawed mental models of onion services that could increase their
susceptibility to phishing attacks.
