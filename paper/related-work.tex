\section{Related work}
\label{sec:related-work}

Tor's user interface has seen numerous substantial changes since its deployment
in 2003~\cite{Syverson2005a}; from a manually-installed Tor ``button,'' to the
Tor Browser Bundle, to the currently-used Tor Browser.

Installation hasn't always been as easy as today, prompting Clark, van Oorschot,
and Adams to use cognitive walkthroughs to study how users install, configure,
and run Tor~\cite{Clark2007a}.  The authors uncovered several usability hurdles
such as jargon-laden documentation, confusing menus, and insufficient visual
feedback.  As of October 2017, the study is ten years old---Tor Browser has
since seen radical changes.

Much more recently, in 2014, Norcie \ea identified stop-points in the
installation and use of the Tor Browser Bundle~\cite{Norcie2014a}.\footnote{The
Tor Browser Bundle was later rebranded and is now known as Tor Browser.}  These
stop-points represent places in a user interface that require action but are met
with confusion by users.  Having identified these stop points, the authors then
issued interface design recommendations and subsequently tested these
recommendations in a user study.

Motivated by Tor's anti-censorship components, Fifield \ea published a design to
study the usability of Tor as a censorship circumvention
tool~\cite{Fifield2015a}.  The authors plan to recruit hundreds of users to
study how they use Tor's configuration wizard in an adversarial setting.  This
effort made use of both qualitative and quantitative methods.  Lee
\ea~\cite{Lee2017a} studied the usability of Tor Launcher, the graphical
configuration tool that allows users to configure Tor Browser.  Their results
paint a bleak picture: 79\% of users' connection attempts in a simulated
censored environment failed.  However, the researchers showed that their
interface improvements resulted in less difficulties for users.

Forte \ea studied the privacy practices of contributors to open collaboration
projects~\cite{Forte2017a}.  The authors interviewed 23 contributors to The Tor
Project and Wikipedia to learn about how privacy concerns affect their
contribution practices.  The study found that contributors worry about an array
of threats, including surveillance, violence, harassment, and loss of
opportunity.

Most recently in 2017, Gallagher \ea conducted a series of semi-structured
interviews to understand both why people use Tor and how they understand the
technology~\cite{Gallagher2017a}.  The authors found that experts tend to have a
network-centric view of Tor and tend to use it frequently over time while
non-experts have a goal-oriented view and see Tor as a black box that provides a
service.  Consequently, non-experts don't use Tor if they don't need its
service.  Furthermore, non-experts tend to consider a single threat while the
threat model of experts contains multiple actors.

We improve on the existing body of work by studying unanswered questions about
the use of Tor Browser and---the focus of this paper---how users interact with
onion services.  We hope that our results generalize to other systems that use
self-authenticating names such as Freenet~\cite{Freenet} and Bitcoin addresses.

While nobody has studied the usability of onion services per se, some research
results are quite related.  To mitigate the usability issue of
randomly-generated domain names, Sai and Fink proposed a mnemonic system that
maps 80-bit onion domains to sentences~\cite{Sai2012a}.  Their work is inspired
by mnemonicode, a method to map binary data to words~\cite{mnemonicode}.
Victors \ea propose a more radical approach by designing the Onion Name
System~\cite{Victors2017a} which allows users to reference an onion service by a
meaningful and globally-unique identifier.  Kadianakis \ea proposed a modular
name system API that allows Tor clients to configure name systems (\eg
GNS~\cite{Schanzenbach2012a} or OnioNS~\cite{Victors2017a}) on a per-domain
basis~\cite{Kadianakis2016a}.  Kadianakis summarized the current state of
research that would improve the naming system behind next-generation onion
domains~\cite{Kadianakis2017a}.
