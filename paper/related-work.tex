\section{Related Work}
\label{sec:related-work}

The usability of Tor Browser has seen numerous substantial changes since its
creation in 2003~\cite{Syverson2005a}; from a manually-installed Tor ``button,''
to the Tor Browser Bundle, and to the currently-used Tor Browser, installation was
not always as easy as it is today. Indeed, the installation woes of yesteryear
prompted Clark, van Oorschot, and Adams 
to use cognitive walkthroughs to study how users install, configure, and run
Tor Browser~\cite{Clark2007a}.  The authors uncovered usability hurdles such as
jargon-laden documentation, confusing menus, and insufficient visual feedback.
As of February 2018, the study is ten years old, and Tor Browser has undergone
radical changes in this time.

More recently, in 2014 Norcie \ea\ identified stop-points in the
installation and use of the Tor Browser Bundle~\cite{Norcie2014a}.\footnote{The
Tor Browser Bundle was later rebranded and is now known as Tor Browser.}  These
stop-points represent places in a user interface that require action but are met
instead with confusion by users.  Having identified these stop-points, the
authors then issued interface design recommendations and subsequently tested
these recommendations in a user study.

Inspired by Tor Browser's use as an anti-censorship system, Fifield \ea\ published
a design to study its usability as a censorship circumvention
tool~\cite{Fifield2015a}.  By drawing on both qualitative and quantitative
methods, the authors plan to recruit hundreds of users to study how they use Tor
Browser's configuration wizard in an adversarial setting.  Lee
\ea~\cite{Lee2017a} studied the usability of Tor Launcher, the graphical
configuration tool that allows users to configure Tor Browser.  Their findings
paint a bleak picture: 79\% of users' connection attempts in a simulated
censored environment failed.  However, the researchers showed that their
proposed interface improvements resulted in less difficulties for users.

Forte \ea\ studied the privacy practices of contributors to open collaboration
projects~\cite{Forte2017a}.  The authors interviewed 23 contributors to The Tor
Project and Wikipedia to learn about how privacy concerns affect their
contribution practices.  The study found that contributors worry about an array
of threats including surveillance, violence, harassment, and loss of
opportunity.

Most recently in 2017, Gallagher \ea\ conducted a series of semi-structured
interviews to understand both why people use Tor Browser and how they understand
the technology~\cite{Gallagher2017a}.  The authors found that experts tend to
have a network-centric view of the Tor network and tend to use it frequently
while non-experts have a goal-oriented view and see Tor Browser as a black box
that provides a service.  Our own work supports these findings.  Consequently,
non-experts do not use Tor Browser if they do not need its service.
Furthermore, non-experts tend to consider a single threat while the threat model
of experts contains multiple actors.

Several research efforts sought to alleviate the handling of randomly-generated
domain names.  Sai and Fink proposed a mnemonic system that maps 80-bit onion
domains to sentences~\cite{Sai2012a}.  Their work is inspired by mnemonicode, a
method to map binary data to words~\cite{mnemonicode}.  Victors \ea\ proposed a
more radical approach by designing the Onion Name System~\cite{Victors2017a}
which allows users to reference an onion service by a meaningful and
globally-unique identifier.  Kadianakis \ea\ designed a modular name system
\textsc{api} that allows Tor clients to configure name systems (\eg,
\textsc{gns}~\cite{Schanzenbach2012a} or OnioNS~\cite{Victors2017a}) on a
per-domain basis~\cite{Kadianakis2016a}.  Kadianakis summarized the current
state of research on onion service naming systems in a blog
post~\cite{Kadianakis2017a}.

We improve on the existing body of work by studying unanswered questions about
the use of Tor Browser and---the focus of this paper---how users interact with
onion services.  We believe that some of our findings generalize to other
systems, \eg, Freenet~\cite{Freenet} and Bitcoin~\cite{Nakamoto2008a} also
employ self-authenticating names---Freenet in its file naming scheme and Bitcoin
in its addressing scheme.
