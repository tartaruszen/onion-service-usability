\section{Results}
\label{sec:results}

\subsection{Data pruning}
\begin{itemize}
    \item Median response time was 15 minutes, but wide variance.
    \item Attention checks didn't work that well.
\end{itemize}

\subsection{Biases}
\begin{itemize}
    \item High-paranoia people didn't respond because our survey required
        JavaScript.
    \item Self-selection bias.
    \item We reached mostly heavy Tor users who presumably know more than the
        occasional user.
    \item Survivor bias: We heard mostly/only from people who can tolerate Tor's
        usability issues.
\end{itemize}

\subsection{Data quality}
\begin{itemize}
    \item Attention checks.
\end{itemize}

\subsection{Demographics}
Table~\ref{tab:survey-demo} shows the demographics of our survey.  Not
surprisingly, our demographic is \emph{young and educated}: more than sixty
percent of respondents are younger than 36, and another sixty percent has at
least a graduate degree.  Finally, another sixty percent considers themselves
at least highly knowledgeable in matters of Internet privacy and security.

% \begin{figure}
% \centering
% \includegraphics[width=\linewidth]{figures/p08-sketch.pdf}
% \caption{Not all subjects had a correct mental model of the Tor network but
% everyone understood that Tor redirects network traffic over nodes.}
% \label{fig:p11-sketch}
% \end{figure}

\begin{table*}[ht]
	\centering
	\begin{tabular}{l r l r l r l r}
	\toprule
	Gender & \% &
	Age & \% &
	Education & \% &
	Domain knowledge & \% \\
	\midrule
	Female & 8.9  & 18--25   & 35.5 & No degree     & 5.5  & No knowledge             & 0.5  \\
	Male   & 86.3 & 26--35   & 34.6 & High school   & 31.9 & Mildly knowledgeable     & 7.6  \\
	Other  & 4.8  & 36--45   & 17.5 & Graduate      & 42.3 & Moderately knowledgeable & 32.4 \\
	       &      & 46--55   & 8.7  & Post graduate & 20.4 & Highly knowledgeable     & 44.9 \\
	       &      & 56--65   & 2.5  &               &      & Expert                   & 14.6 \\
	       &      & $>$ 65   & 1.2  &               &      & & \\
	\bottomrule
	\end{tabular}
	\caption{The distribution over gender, age, education, and domain knowledge 
	for our 621 interview subjects.}
	\label{tab:survey-demo}
\end{table*}

\subsection{Tor usage}
The three entities most Tor users seek to protect themselves against are
governments (X\%), ISPs (X\%), and corporations (X\%).  Generally speaking,
people use Tor to protect themselves from a wide variety of threats including:
\begin{description}
    \item[Corporations] because of tracking, advertising, and profiling.  Some
        respondents specifically pointed out Google and Facebook.
    \item[Service providers] such as ISPs, backbone ISPs, and websites
        themselves.  In fact, the ISP is the most popular ``threat'' to our
        respondents with XX\%.
    \item[Governments] because of mass surveillance, law enforcement, and
        government-mandated censorship.  Governments are the second most
        popular threat with XX\%.
    \item[Personal threats] including identity theft, targeted harassment, and
        stalking.
    \item[Research] allows users to learn about a topic without revealing their
        interest in it.  Some respondents use Tor for search engine
        optimization, computer security research, and to research medical
        conditions.
    \item[Technical] use cases include IPv6 connectivity, the evasion of
        geographical restrictions, and access to onion services.  A small
        number of respondents is only interested in technical aspects other
        than privacy.
\end{description}

One respondent stated that they don't need anonymity themselves but instead use
Tor to provide cover traffic for ``people who need protection.''

Another respondent uses Tor to have IPv6 connectivity---some exit relays are
multi-homed and can connect to both IPv4 and IPv6 endpoints.

Finally, some respondents mentioned that they are using Tor only to connect to
onion services.

Half of our respondents use Tor either once a day or consider Tor Browser their
main browser.

\subsection{Onion service usage}

\subsubsection{Perception of next-generation domain format}

Recall that the next generation of onion services will extend the domain length
from 16 to 54 characters, almost three times the length of the current domain
format.

We asked our survey respondents if they expect this format change to affect
their browsing habits.  591 users answered this question.  95 (16\%) selected
that prop224 domains will change their habits while the remaining 496 (84\%)
selected that their habits won't be affected.

Respondents who believe that their habits will change (16\%) gave the
following reasons:

These results suggest that the new domain format is among the
minor usability issues surrounding onion services.  In fact, an
easy-to-remember domain format ranks last among the six criteria whose
importance we asked users about.  On a five-point Likert scale ranging
from "not at all important" to "very important," we got the following
results:

Problematic:
``I only memorize the first part of the domain''

``If there isn't some cognizable word at the start, it'll be more difficult for
me to determine if I'm going to the correct domain or a scam. I may end up going
to less .onion sites as a result.''

\subsubsection{Susceptibility to phishing}
\begin{itemize}
    \item Attack partially documented in academic
        literature~\cite[\S~5.1]{Winter2016a}.  Monteiro provided a
        timeline~\cite{Monteiro2016a}.
    \item Many ad-hoc anti-phishing methods, some of which incorrect.
\end{itemize}

\subsubsection{Perception of vanity onion domains}
\begin{description}
    \item[$+$] Combination of vanity prefix and Markov Model-selected suffix can
        deliver memorable domain for v2 domains.
    \item[$+$] Reveals what the onion service is hosting.
    \item[$-$] May provide false sense of security.
    \item[$-$] Only available to people with resources.
    \item[$-$] Conflicts with peoples' (flawed) expectation that a domain must
        be random to stay hidden.
\end{description}

Technical ways to exacerbate creation of vanity domains: Incorporate scrypt into
key generation process.
If vanity domains remain, have Tor process support it?

\subsection{Onion service operation}
\subsection{Expectations of privacy}

\subsection{Participants}
\begin{itemize}
    \item Provide intuition on how much our sample resembles the general
        population.
    \item From when to when did we disseminate our survey?
    \item How many participants did we attract?
    \item How long did it take to complete the survey?  Provide some descriptive statistics.
    \item What's the female/male ratio?
    \item What's the age distribution?
    \item What's the level of education?
    \item What's the computer security knowledge?  Note that our participants
        may overestimate their knowledge.
\end{itemize}

\subsection{Random anecdotes}
\begin{itemize}
    \item In pre-test: one person considered the onion domain itself
        confidential and was worried about the domain's prefix revealing
        anything about its content.
\end{itemize}
