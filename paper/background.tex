\section{Background}
\label{sec:background}

Onion services are TCP services that are only accessible over the Tor network.
While conventional Internet services are contacted over their IP address, onion
services are contacted over their onion domain, which is resolved and routed
inside the Tor network.  Due to onion services being hosted in the Tor network,
network traffic traverses six Tor relays (see Figure~\ref{fig:onion-service})
before it reaches the onion service.  The resulting increase in latency is the
cost of the anonymity that onion services provide.

\begin{figure*}[ht]
\centering
\begin{tikzpicture}[node distance=0.5cm]
\tikzset{>=latex}

\tikzstyle{block} = [rectangle, draw, rounded corners, text centered,
                     minimum height=0.5cm]

\node[block,fill=green!20]             (TB)  {Tor Browser};
\node[block,fill=blue!20,right=of TB]  (GR1) {Guard};
\node[block,fill=blue!20,right=of GR1] (MR1) {Middle};
\node[block,fill=gray!20,right=of MR1] (R)   {Rendezvous};
\node[block,fill=blue!20,right=of R]   (MR2) {Middle};
\node[block,fill=blue!20,right=of MR2] (MR3) {Middle};
\node[block,fill=blue!20,right=of MR3] (GR2) {Guard};
\node[block,fill=red!20,right=of GR2]  (OS)  {Onion service};

\draw[<->] (TB.east)  -- (GR1.west);
\draw[<->] (GR1.east) -- (MR1.west);
\draw[<->] (MR1.east) -- (R.west);
\draw[<->] (R.east)   -- (MR2.west);
\draw[<->] (MR2.east) -- (MR3.west);
\draw[<->] (MR3.east) -- (GR2.west);
\draw[<->] (GR2.east) -- (OS.west);

\end{tikzpicture}
\caption{When a user connects to an onion service, there are six relays in
between her Tor Browser (on the left) and the onion service (on the right).}
\label{fig:onion-service}
\end{figure*}

To create an onion domain, a Tor client first generates an RSA key pair.  It
then computes the SHA-1 hash over the RSA public key, truncates it to 80 bits,
and encodes these 80 bits in Base32, resulting in sixteen characters, \eg,
\texttt{expyuzz4wqqyqhjn}.  Due to the domain being a function of the public
key, onion domains are self-authenticating, meaning that as long as a client has
the correct domain, it knows what public key to expect.  The downside is that
sixteen random characters are impractical to remember.  However, onion domains
can be made at least partially meaningful by repeatedly creating RSA keys until
the resulting domain contains a desired string.  These domains are referred to
as \emph{vanity onion domains}.  The longer the desired string, the more time it
takes to find a matching key pair.  In practice, onion service operators use
tools such as scallion~\cite{scallion} that parallelize the search for suitable
keys to speed up the process.  Vanity domains are used by several organizations
such as Facebook (\url{facebookcorewwwi.onion}), ProPublica
(\url{propub3r6espa33w.onion}), and the New York Times
(\url{nytimes3xbfgragh.onion}).

Onion services are private by default.  Once an onion service is created, it is
up to its operator to announce it to the public, \eg, by adding it to onion site
search engines such as Ahmia.\footnote{The search engine is available online at
\url{https://ahmia.fi}.}  The lack of a go-to service such as Google for
onion service discovery means that the community has devised various ways to
publish onion services; most importantly an array of search engines and curated
lists.
