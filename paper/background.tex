\section{Background}
\label{sec:background}

Onion services are \textsc{tcp} services that are only accessible over the Tor
network.  Onion domains take the place of \textsc{ip} addresses, which
traditional \textsc{tcp} services use for addressing.  These onion domains are
only meaningful inside the Tor network, and help in establishing a circuit
between the client and the onion service which (by default) features six hops.

The creation of a new onion domain requires a Tor daemon to generate an
\textsc{rsa} key pair.  It then computes the \textsc{sha}-1 hash over the
\textsc{rsa} public key, truncates it to 80 bits, and encodes these 80 bits in
Base32, resulting in sixteen characters, \eg, expyuzz4wqqyqhjn.  As of February
2018, The Tor Project is deploying the next generation of onion services whose
domain format will feature 56 instead of sixteen
characters~\cite[\S~6]{Mathewson2013a}---a Base32 encoding of the onion
service's public key, a checksum, and a version number.  Because of the next
generation using elliptic curve cryptography, the entire public key (instead of
just a hash over the public key) is embedded in the domain.

Due to an onion domain's being a function of its public key, onion domains are
self-authenticating, \ie, as long as a client has the correct domain, it knows
what public key to expect.  The downside is that sixteen random characters are
impractical to remember, let alone 56 characters.

We can make onion domains at least partially meaningful by repeatedly creating
\textsc{rsa} keys until the resulting domain contains a desired string.  We call
these \emph{vanity onion domains}.  A vanity prefix of length $n$ takes on
average $0.5 \cdot 32^n$ key creations given Base32's alphabet size of 32
characters.  After having created a set of domains featuring a vanity prefix,
one can search this set for the domain that is the easiest to remember, \eg, by
using a Markov model to filter domains that resemble words in the English
language.  While this method will not produce fully-meaningful domains, it can
facilitate memorization as evidenced by the vanity domains of Facebook
(facebookcorewwwi.onion), ProPublica (propub3r6espa33w.onion), and the New York
Times (nytimes3xbfgragh.onion).  In practice, many onion service operators use
the tool scallion~\cite{scallion} to parallelize the search for vanity domains.

Onion services are private by default.  Once an onion service is created, it is
its operator's task to disseminate the domain, \eg, by adding it to onion site
search engines such as Ahmia~\cite{ahmia}.  The lack of a go-to service such as
a ``Google for onion services'' prompted the community to devise various ways to
disseminate onion services, most importantly an array of search engines and
curated lists.
