\section{Interview study}
\label{sec:interview-study}

We conducted a series of semi-structured interviews about the use of Tor
Browser and onion services to explore user experience outside the constraints
of our online survey, and to inform the creation of our survey---and vice versa
because we conducted more interviews after our survey ended.

\subsection{Method}

We developed a question set that served as the basis for each interview.  The
semi-structured nature of the interviews allowed us to deviate from our
questions, \eg, by asking follow-up questions.  Our question set began with
demographic information (gender, age range, occupation, country of residence,
and level of education), followed by information about online behavior, and
finally questions specific to Tor Browser and onion services.

\subsubsection{Recruitment}

We created a short pre-interview survey (see
Appendix~\ref{sec:interview-survey}) to select eligible subjects.  This survey
was advertised in a post on The Tor Project's blog~\cite{Winter2017a} and its
Twitter account.  Our selection process focused on laypeople and sought to
maximize cultural, gender, location, education, and age diversity.  We ended up
interviewing seventeen subjects whose demographic information is shown in
Table~\ref{tab:interviewees}.  In addition to our online screening, we
recruited participants in person at an Internet freedom event.

It is difficult to draw a uniform sample of Tor users to interview. We belive
that The Tor Project's blog and Twitter account are mainly read by users who
are particularly technical while many non-technical users may install Tor
Browser in a one-off process and then never read anything again.  To make
matters worse, due to Tor Browser's very nature as an anonymity tool, many
users value their privacy significantly more than the average Internet user,
making it difficult to have users trust us enough to talk about their browsing
habits.  We believe that our sample is biased towards more knowledgable and
technical users, but it also shows how diverse Tor's user base is.  Our
participants were human rights activists, legal professionals, writers,
artists, and journalists, just to name a few.

\begin{table*}[ht]
    \centering
    \begin{tabular}{l r r l r r l r r l r r}
    \toprule
    Age & \# & \% &
    Gender & \# & \% &
    Country of residence & \# & \% &
    Education & \# & \% \\
    \midrule
    18--25 & 2  & 11.8 & Female & 5  & 29.4 & Australia      & 1 & 5.9  & No degree     & 1  & 5.9 \\
    26--35 & 10 & 58.8 & Male   & 12 & 70.6 & Canada         & 5 & 29.4 & High school   & 3  & 17.7 \\
    36--45 & 4  & 23.5 &        &    &      & Colombia       & 1 & 5.9  & Graduate      & 3  & 17.7 \\
    46--55 & 1  & 5.9  &        &    &      & Germany        & 1 & 5.9  & Post graduate & 10 & 58.8 \\
           &    &      &        &    &      & India          & 1 & 5.9  & & & \\
           &    &      &        &    &      & Indonesia      & 1 & 5.9  & & & \\
           &    &      &        &    &      & Mexico         & 1 & 5.9  & & & \\
           &    &      &        &    &      & Netherlands    & 1 & 5.9  & & & \\
           &    &      &        &    &      & South Korea    & 1 & 5.9  & & & \\
           &    &      &        &    &      & United Kingdom & 2 & 11.8 & & & \\
           &    &      &        &    &      & U.S.A.         & 2 & 11.8 & & & \\
    \bottomrule
    \end{tabular}
    \caption{The distribution over gender, age, country of residence, and
    education for our seventeen interview subjects.}
    \label{tab:interviewees}
\end{table*}

\subsubsection{Procedure}

We conducted thirteen interviews in person and four interviews remotely; over
Skype, Signal, WhatsApp, and Jitsi---depending on what our interviewees felt
the most comfortable with.  For in-person interviews we asked our interviewees
to sign a consent form.  This was not practical for remote interviews, so we
sent the consent form in advance over email and asked for verbal consent before
the interview.  In all cases we explicitly asked for permission to record the
conversation.  All except two participants agreed to have the interview
recorded.  For these two interviews we took notes instead.  All interviews
concluded with a debriefing phase in which we asked if our participants had any
remaining questions.  Some were curious if technical explanations they provided
earlier were correct.  Finally, we offered our participants a gift card worth
20 USD as a token of appreciation.

Once all interviews were transcribed, we employed qualitative data coding to
analyze the transcripts.  Each interview transcript was coded by two members of
our team.  This process identified TODO themes, all listed in
Appendix~\ref{sec:coding-themes}.

\subsection{Research ethics}

The institutional review board (IRB) of Princeton University deemed our study
exempt from further review.\footnote{Our IRB protocol number is 8251.}  Before
each in-person interview we asked our subjects to sign a consent form.  This
was not practical for remote interviews, so we sought permission from our IRB
to use verbal instead of written consent.  In all interviews we explicitly
asked for permission to record the conversation, to which all but two
participants agreed.  In these two cases the interviewer took notes instead.
Finally, we made it clear that our participants could withdraw their consent at
any time.  Once we transcribed our interview recordings, we deleted the
original recordings.  We used the services of the company Rev to have our
recordings transcribed.  A mutually-signed non-disclosure agreement protected
the confidentiality of our data.

\subsection{Results}

\subsubsection{Advantages}

\begin{itemize}
    \item Better privacy and security
    \item Gives you control over your data
    \item Perceived feeling of safety
\end{itemize}

\subsubsection{Disadvantages}

\begin{itemize}
    \item Slow browsing
    \item 90s browsing experience
    \item CAPTCHAs
    \item Tor makes you stick out
    \item Harmful ``public image'' of Tor
\end{itemize}

\subsubsection{Threat model}

\begin{itemize}
    \item Advertising companies
\end{itemize}

\subsubsection{Onion services}

\begin{itemize}
    \item Some users feel safer, some less safe when using onion services
\end{itemize}

\subsubsection{Miscellaneous}

\begin{itemize}
    \item Briding cultural gaps
\end{itemize}
